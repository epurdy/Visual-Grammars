
We use a very straightforward data model, that of a Bernoulli model
over Canny edges.

We set $P(E_p=1 \mid p\in O) = p_{fg}$, $P(E_p=1 \mid p\notin O) = p_{bg}$.

We use ratio of foreground to background, so that we only have to
multiply over pixels in the object.

\begin{align*}
P(E\mid O) &= \prod_{p\in O} p_{fg}^{E_p} (1-p_{fg})^{1-E_p} \cdot
\prod_{p\notin O} p_{bg}^{E_p} (1-p_{bg})^{1-E_p}\\
&= \prod_{p\in O} \left(\frac{p_{fg}}{p_{bg}}\right)^{E_p}
\left(\frac{1-p_{fg}}{1-p_{bg}}\right)^{1-E_p} \cdot
  \prod_{p} p_{bg}^{E_p} (1-p_{bg})^{1-E_p}\\
&\propto \prod_{p\in O} \left(\frac{p_{fg}}{p_{bg}}\right)^{E_p}
\left(\frac{1-p_{fg}}{1-p_{bg}}\right)^{1-E_p}\\
\log P(E\mid O)&= \sum_{p\in O}\left[ E_p \log \left(\frac{p_{fg}}{p_{bg}}\right)
+ (1-E_p) \log\left(\frac{1-p_{fg}}{1-p_{bg}}\right)\right] + C\\
&= \sum_{p\in O}\left[ E_p \log \left(\frac{p_{fg}(1-p_{bg})}{p_{bg}(1-p_{fg})}\right)
+ \log\left(\frac{1-p_{fg}}{1-p_{bg}}\right)\right] + C\\
&= \sum_{p\in O}\left[ E_p \log \left(\frac{p_{fg}(1-p_{bg})}{p_{bg}(1-p_{fg})}\right)\right]
+ |O|\log\left(\frac{1-p_{fg}}{1-p_{bg}}\right) + C
\end{align*}

When doing scene parsing, we will modify this formula slightly. Pixels
that are in already selected objects will be labelled foreground, and
other pixels will initially be labelled background. Let $F$ be the set
of pixels already marked foreground.

\begin{align*}
\log P(E\mid O) &= \sum_{p\in O\sm F}\left[ E_p \log \left(\frac{p_{fg}(1-p_{bg})}{p_{bg}(1-p_{fg})}\right)\right]
+ |O\sm F|\log\left(\frac{1-p_{fg}}{1-p_{bg}}\right) + C
\end{align*}

For pixels in $F\cap O$, the ratio is one, and thus its log is
zero. For pixels not in $O$, the ratio becomes a constant as before.
