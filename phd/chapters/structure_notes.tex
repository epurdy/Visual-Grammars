\section{Merging as a Correspondence Problem}

Devi Parikh got some decent results using the approach from ``A
Spectral Technique for Correspondence Problems Using Pairwise
Constraints'' by Marius Leordeanu and Martial Hebert. In order to use
their framework, we have to have similarity terms and context terms.

Their framework requires that these terms are nonnegative. 

Their framework can produce correspondences that obey mapping
constraints like one-to-oneness. We might not care about that.

\bitem
\item Similarity term. What is the reward for merging $X$ and $Y$? 
  The KL divergence actually works pretty well.

  If we think of $X$ as stealing the data $d_Y$ that $Y$ was
  explaining, then we'll be paying $\log
  \frac{P_Y(d_Y)}{P_X(d_Y)}$. If we don't know $d_Y$, then we'll just
  have to guess, so we'll be paying $E_Y \left[ \log
    \frac{P_Y(d_Y)}{P_X(d_Y)} \right]$, which is exactly the KL
  divergence.

  If the formula were correct, this would be weighted by the actual
  amount of data that $Y$ is explaining, but we can maybe assume that
  this is uniform pre-training. Or weight the KL divergence term by
  how much data we think $Y$ should explaining.

\item Context term. What is the reward for simultaneously merging
  $X_1$ with $X_2$ and $Y_1$ with $Y_2$? Probably just a bonus if they
  are neighbors in some well-defined sense.

  One such sense: $X_1 Y_1$ and $X_2 Y_2$ are both right-hand
  sides. Or $X_1 \to Y_1 Z_1$ and $X_2 \to Y_2 Z_2$ are rules.

  It might be good to generalize this to other more subtle kinds of
  neighborship. For example, the rightmost descendants of $B$ in $A\to
  BC$ will be always be next to the leftmost descendants of $C$.

  The context term can be interpreted as approximating the benefit
  from merging two rules and no longer having to encode one of them,
  or pay its Dirichlet/midpoint prior cost. To actually get it right,
  you'd have to have ternary terms, and there's no obvious way to do
  that.

  \eitem

\section{Building small grammars with choice}

We have a grammar $\GGG$, and we are recursively using it to pick
midpoints of a curve. We will say that $\GGG$ is $\theta$-complete if
the midpoint choice of a subcurve of length $L$ can be made within
$\theta L$ points of any midpoint choice.

\begin{thm}
Let $C$ be a curve of length $N = 2^n$. Then there is a
$\frac{1}{2^{t+1}}$-complete grammar \GGG\ that has at most
$(2^{3t} + 2^{2t}) N$ rules. (So, for $\theta=\frac{1}{8}$,
\GGG\ has at most $80 N$ rules.)
\end{thm}

\begin{proof}
We represent subcurves of $C$ by intervals of indices of the form
$[x,y]$, where $0\le x < y \le N$.

We make the following definitions:
$$k_{max} = \lfloor \frac{n}{t} \rfloor$$
$$\LLL_k = \{ [x,y] : 2^{kt} | x, 2^{kt} | y \} \mbox{ for } 0 \le k \le k_{max}.$$ 

$$\mbox{(Note that } \LLL_0 \supset \LLL_1 \supset \cdots \supset \LLL_{k_{max}}.)$$

$$\AAA_k = \{ [x,y] \in \LLL_k : (y-x) \le 2^{(k+2)t}\}$$

$$\NNN_k = \{ [x,y] \in \AAA_k : [x,y] \notin \AAA_j \mbox{ for } j
< i \}$$

$$\RRR_k = \{ [x,y] \to [x, z] [z, y] : [x,y] \in
\NNN_k, z = x + a2^{kt}, x < z < y \}\mbox{ for } 1\le k \le k_{max}$$

We define $\GGG$ to have rules $\RRR$ and nonterminals $\NNN$, where
$\NNN = \cup_{k=0}^{k_{max}} \NNN_k$ and
 $\RRR = \cup_{k=1}^{k_{max}} \RRR_k$.

First, we claim that for $[x,y]\to [x,z][z,y]\in \RRR_k$, $[x,z]$ and
$[z,y]$ are in $\AAA_k$. This holds because:
$$[x,y]\in \NNN_k \implies 2^{kt} | x,y \implies 2^{kt} | z = x + 2^{kt}a$$
and
$$[x,y]\in \NNN_k \implies (y-x) \le 2^{(k+2)t},$$ and since $[x,z],
[z,y]$ are shorter than $[x,y]$, the same applies to them.

Second, we claim that every subcurve $[x,y]\in \NNN_k$ has at least
$2^t - 1$ evenly spaced choices of $z$ such that $([x,y] \to [x,z]
[z,y]) \in \RRR_k$. This is true if $(y-x) \ge 2^t \cdot 2^{kt} =
2^{(k+1)t}$, which is true, because $[x,y] \in \LLL_{k-1}$ and
$[x,y]\notin \AAA_{k-1}$.

Since we have at least $2^t - 1$ evenly spaced choices of midpoint, we
can choose the midpoint within $2^{t+1} (y-x)$ of any desired
midpoint.

Now we must bound the size of $\RRR$.
\begin{align*}
|\RRR| &= \sum_{k=1}^{k_{max}} |\RRR_k| \\
&\le \sum_{k=1}^{k_{max}} |\NNN_k| \cdot \frac{\max_{[x,y]\in\NNN_k} y-x }{2^{kt}}\\
&\le \sum_{k=1}^{k_{max}} |\NNN_k| \cdot 2^{(k+2)t} / 2^{kt}\\
&\le \sum_{k=1}^{k_{max}} |\AAA_k| 2^{2t}\\
&\le \sum_{k=1}^{k_{max}} 2^{n - kt} \cdot (2^{2t} - 1) \cdot 2^{2t}\\
&= 2^{n+2t}(2^{2t}-1) \sum_{k=1}^{k_{max}} 2^{-kt}\\
&= 2^{n+2t}(2^{2t}-1) \frac{2^{-t}}{1 - 2^{-t}}\\
&= 2^{n+2t}(2^{t}-1)(2^t + 1) \frac{1}{2^t - 1}\\
&= 2^{n+2t}(2^t + 1)\\
&= (2^{3t} + 2^{2t}) N
\end{align*}

\end{proof}

\section{Computing the KL divergence}

Consider the distribution $P_X$ of curves arising from a nonterminal
$X$.  We wish to compute the KL divergence between $P_X$ and $P_Y$, so
that we know whether we want to merge them.

This will have to be highly approximate.

\subsection{Dealing with Mixture Distributions}

The first problem with computing the KL divergence between $P_X$ and
$P_Y$ is that these are mixture distributions, with components of the
form $P_{X\to AB}$, $P_{Y\to CD}$ for each rule we could use to expand
$X$ or $Y$, respectively.

I found a paper: ``APPROXIMATING THE KULLBACK LEIBLER DIVERGENCE
BETWEEN GAUSSIAN MIXTURE MODELS'' by John R. Hershey and Peder
A. Olsen. As the title suggests, they compare a whole bunch of
approximations of the KL divergence between Gaussian mixture
models. Since the divergence between Gaussians has a closed form, this
problem is much easier than ours, where we have a hierarchical mixture
model. Nevertheless, it seems like a good starting point.

The most accurate approximation is, of course, a Monte Carlo estimate
with a huge number of samples. They look at many methods, but for us
the relevant ones seem to be:

\bitem
\item Monte Carlo with fewer samples. 
\item Two eldritch variational horrors.
\item One approximation from the paper ``An efficient image similarity
  measure based on approximations of KL-divergence between two
  gaussian mixtures'', by Jacob Goldberger, Shiri Gordon, and Hayit
  Greenspan.

  According to Hershey and Olsen, this is reported to work poorly for
  GMM's with a few low-probability components. That particular
  objection doesn't apply to us, since our components are uniform.
\eitem

Anyway, the GGG idea is this:

\begin{align*}
KL(f || g) &= \sum_{i=1}^n \alpha_i \int f_i \log f - \sum_{i=1}^n \alpha_i \int f_i \log g\\
&\approx \sum_{i=1}^n \alpha_i \int f_i \log \alpha_i f_i - \sum_{i=1}^n \alpha_i \max_j \int f_i \log \beta_j g_j\\
&= \sum_{i=1}^n \alpha_i \min_j \left( KL(f_i || g_j) + \log \frac{\alpha_i}{\beta_j} \right)
\end{align*}

In our context, this last formula is very simple, since we are talking
about doing merges before any training, and the $\alpha_i, \beta_j$
will have uniform values. This approximation then tells us to do an
essentially CKY-like process to compute the KL divergence of mixtures.

Who knows how well it would work for us?

\subsection{Dealing with Binary Rules}

Now we need to figure out $KL( P_{X\to X_1 X_2} || P_{Y \to Y_1
  Y_2})$. As Pedro said, this gets really messy if we do it correctly,
and allow $X$ and $Y$ to use different midpoints in their parses, or
even get a full parse score instead of Viterbi.

However, if we don't do that, but instead limit $Y$ to parsing an
$X$-generated curve in the same way that $X$ generated it, then the
problem might not be intractable.

Let $C$ be the $X$-generated curve, where $C=DE$ is the decomposition
specified by $X$. Then, for our purposes, $P_{X\to X_1 X_2}$ is just a
product distribution over $D$, $E$, and the choice of the midpoint.

The KL divergence is additive over independent distributions, so we get
$$KL(P_{X\to X_1 X_2}) || P_{Y\to Y_1 Y_2} = KL(P_{X_1} || P_{Y_1}) +
KL(P_{X_2} || P_{Y_2}) + KL(\mu_{X\to X_1 X_2} || \mu_{Y\to Y_1
  Y_2}).$$

This last term is just the KL divergence of the midpoint
distributions. This will either be the KL divergence of two Watson
distributions (which might have a closed form?) or the KL divergence
of two Gaussian mixtures for a non-parametric midpoint model. In the
second case, we can use the GGG approximation with more confidence, or
use some other approximation from Hershey and Olsen.

