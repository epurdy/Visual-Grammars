\marginnote{Beginning of learning/doing\_em.tex}

\section{EM}
\marginnote{We want pictures of a grammar that does visible learning
  in a single iteration. The simplified hand is a nice
  example. Perturb the midpoint distributions and then try to recover
  them by retraining.}

\marginnote{Pictures to represent what EM does generically? Not clear
  how much insight it provides.}

Let $C_1,\dots, C_n$ be independent samples from a grammar $\GGG$, for
which we know the structure, but not the parameters. We would like to
set the parameters of $\GGG$ such that the posterior $P(\GGG\mid
C_1,\dots,C_n)$ is maximized. We can write the posterior as
\begin{align*}
  P(\GGG\mid C_1,\dots, C_n) &= P(\GGG) \cdot \frac{\prod_i P(C_i\mid \GGG)}{\prod_i P(C_i)}\\
  &\propto P(\GGG) \cdot \prod_i \sum_{T\in Parse_\GGG(C_i)}P(C_i, T \mid \GGG)\\
  &= P(\GGG) \cdot \sum_{\{T_i\}}\prod_i P(C_i, T_i \mid \GGG)\\
\end{align*}
Unfortunately, it is not known how to maximize this posterior, or even
the likelihood. The likelihood is known to have many local maxima in
the case of context-free grammars for natural languages
\cite{charniak}. Therefore, we are forced to use approximation
techniques. We use the Expectation-Maximization algorithm \cite{em},
which is a standard approach to finding maximum likelihood or maximum
a posteriori estimates in the case where important variables are
unobserved. In our case, the unobserved variables are the parse trees
$\{T_i\}$.

Let $\mathbf{C} = (C_1,\dots,C_n)$, and let $\mathbf{T}$ be the latent
variable $(T_1,\dots,T_n)$. We then iteratively find better and better
grammars $\GGG^{(t)}$:
\begin{enumerate}
\item \textbf{E step:} Let 
  \begin{align*}
Q^{(t)}(\GGG) &= \sum_{\mathbf{T}} \big[P(\mathbf{T}\mid \mathbf{C},
\GGG^{(t)}) \log P(\mathbf{C},\mathbf{T}\mid \GGG) \big] + \log
P(\GGG)\\
&= E_{\mathbf{T} \sim \mathbf{T} | \mathbf{C}, \GGG^{(t)}}\big[\log
P(\mathbf{C},\mathbf{T}\mid \GGG) \big] + \log P(\GGG)
  \end{align*}
\item \textbf{M step:} Let
$$ \GGG^{(t+1)} = \argmax_{\GGG} Q^{(t)}(\GGG)$$
\end{enumerate}

\subsection{The E Step}

$$Q^{(t)}(\GGG) = E_{\mathbf{T}\sim \mathbf{T}, \GGG^{(t)}} [\log
P(C,T\mid \GGG)] + \log P(\GGG)$$
Since $P(\mathbf{C},\mathbf{T}\mid \GGG) = \prod_a P(C_a, T_a\mid \GGG),$
and by linearity of expectation:
$$Q^{(t)}(\GGG) = \sum_a E_{T_a\sim \Delta^{(t)}_a} [\log
P(C_a,T_a \mid \GGG)] + \log P(\GGG),$$
where $\Delta^{(t)}_a(T_a)$ is the distribution $P(T_a\mid C_a,
\GGG^{(t)})$.

Let $Q_a^{(t)}(\GGG) =E_{T_a \sim \Delta_a^{(t)}}[\log P(C_a,T_a\mid
\GGG)].$ We can decompose $P(C_a, T_a\mid \GGG)$ by using Proposition
\ref{prop-inside-corr} \footnote{If $\GGG$ is a grammar for closed curves,
  $P(C_a,T_a\mid \GGG)$ will have an extra factor of
  $\frac{1}{|C|}$. We can ignore this because the logarithm turns it
  into an additive constant, which is irrelevant in the M step.}:

\begin{align*}
  Q_a^{(t)}(\GGG) &=
E_{T_a \sim \Delta_a^{(t)}}\left[\log
    \prod_{\langle X_{ij} \to \lambda\rangle \in T_a} P_{sub}(\langle
    X_{ij} \to \lambda\rangle)\right]\\
&=
E_{T_a \sim \Delta_a^{(t)}}\left[
    \sum_{\langle X_{ij} \to \lambda\rangle \in T_a} \log P_{sub}(\langle
    X_{ij} \to \lambda\rangle)\right]\\
\intertext{
We can rewrite this as a sum over all possible concrete
  substitution rules, using indicator variables to eliminate absent terms:}
&=
E_{T_a \sim \Delta_a^{(t)}}\Bigg[
\sum_{i < j < k} \sum_{[X\to YZ]\in \RRR}\ind [ 
\langle X_{ik}
\to Y_{ij} Z_{jk} \rangle] \quad \log P_{sub}(\langle X_{ik} \to Y_{ij}
Z_{jk}\rangle )\\
&\qquad+\qquad \sum_{i} \sum_{[X\to \ell]\in \RRR}\ind [ \langle X_{i i+1}
\to \ell_{i i+1} \rangle] \quad \log P_{sub}(\langle X_{i i+1} \to \ell_{i i+1}\rangle)
\Bigg]\\
&=
\sum_{i < j < k} \sum_{[X\to YZ]\in \RRR} \log P_{sub}(\langle X_{ik} \to Y_{ij}
Z_{jk}\rangle ) \quad E_{T_a \sim \Delta_a^{(t)}}[\ind [ 
\langle X_{ik}
\to Y_{ij} Z_{jk} \rangle]]\\
&+ \sum_{i} \sum_{[X\to \ell]\in \RRR}
\log P_{sub}(\langle X_{i i+1} \to \ell_{i i+1}\rangle)
\quad E_{T_a \sim \Delta_a^{(t)}}[\ind [ \langle X_{i i+1}
\to \ell_{i i+1} \rangle]] 
\end{align*}
In the case of closed curves, the sum over $i < j < k$ will instead be
over $0\le i,k \le n$, and $j\in cw_n(i,k)$.

\marginnote{Pick a particular rule $X\to YZ$ and show the heaviest few
  values over $a,i,j,k$. This should give a lot of insight into the
  soft counts.}

Let 
\begin{align*}
  Count_a^{(t)}(\langle X_{ik} \to Y_{ij} Z_{jk} \rangle) &=  E_{T_a \sim \Delta_a^{(t)}}[\ind [ 
\langle X_{ik}
\to Y_{ij} Z_{jk} \rangle]]\\
&= P_{T_a \sim \Delta_a^{(t)}}( \langle X_{ik} \to Y_{ij} Z_{jk}
\rangle \in T_a)\\
Count_a^{(t)}(\langle X_{ii+1} \to \ell_{i i+1} \rangle) &=
E_{T_a \sim \Delta_a^{(t)}}[\ind [ \langle X_{i i+1}
\to \ell_{i i+1} \rangle]] \\
&= P_{T_a \sim \Delta_a^{(t)}}( \langle X_{i i+1} \to \ell_{i i+1} \rangle \in T_a)\\
\end{align*}
and let
\begin{align*}
  Q^{(t)}_{a,i,j,k,[X\to YZ]}(\GGG) &= 
\log P_{sub}(\langle X_{ik} \to Y_{ij}
Z_{jk}\rangle ) \cdot   Count_a^{(t)}(\langle X_{ik} \to Y_{ij} Z_{jk} \rangle)\\
Q^{(t)}_{a,i,[X\to \ell]}(\GGG) &=
\log P_{sub}(\langle X_{i i+1} \to \ell_{i i+1}\rangle)
\cdot Count_a^{(t)}(\langle X_{ii+1} \to \ell_{i i+1} \rangle).
\end{align*}

\subsection{Computing the Soft Counts}
In this section, we show how to compute the soft counts from the inside and outside weights.

Firstly,
\begin{align*}
\Delta_a^{(t)} &= P(T_a \mid C_a, \GGG^{(t)})\\
 &= \frac{P(C_a, T_a \mid \GGG^{(t)})}{P(C_a \mid \GGG^{(t)})}\\
&= \frac{P(T_a \mid \GGG^{(t)})}{P(C_a \mid \GGG^{(t)})}.\\
\end{align*}

\begin{defn}
A {\em complete parse tree} is a parse tree with root node $S_{0n}$,
and which has no unexpanded nodes. We will denote the set of complete
$\GGG$-parse trees of $C$ by $\FF_\GGG^C$. We will denote the set of
complete $\GGG$-parse trees of $C$ which contain a node $\langle
X_{ij}\to \lambda \rangle$ by $\FF_\GGG^C(X_{ij}, \lambda)$.
\end{defn}

\begin{prop}
Every parse tree $T \in \FF_\GGG^C(X_{ij}, \lambda)$ can be obtained as
$$T_{out} \triangle^{\cancel{X_{ij}}}_{(X_{ij},\lambda)} T_{in},$$ 
where $T_{out}\in \Outer_\GGG^C(X_{ij})$ and $T_{in} \in
\Inner_\GGG^C(X_{ij}, \lambda)$.
\end{prop}
\begin{proof}
If we remove the subtree rooted at the node $\langle X_{ij} \to
\lambda \rangle$ and replace it with an unexpanded node $X_{ij}$, the
resulting tree is an outer parse tree in $\Outer_\GGG^C(X_{ij})$. The
subtree rooted at $\langle X_{ij} \to \lambda\rangle$ is an inner
parse tree with root $\langle X_{ij} \to \lambda \rangle$.
\end{proof}

\begin{obs}
For open curves $C$,
\begin{align*}
P(\langle X_{ij} \to \lambda \rangle \in T \mid \GGG) &=
\sum_{T \in \FF_\GGG^C(X_{ij}, \lambda)} W_\GGG(T)\\
\intertext{which, by the previous proposition can be broken up as}
&=
\left(\sum_{T_{out}\in \Outer_\GGG^C(X_{ij})}
  W_\GGG(T_{out}) \right)
\left(\sum_{T_{in}\in \Inner_\GGG^C(X_{ij},\lambda)} W_\GGG(T_{in})
\right)\\
&= Outside_\GGG^C(X_{ij}) Inside_\GGG^C(X_{ij}, \lambda).
\end{align*}
For closed curves $C$, 
\begin{align*}
P(\langle X_{ij} \to \lambda \rangle \in T \mid \GGG) &=
\frac{1}{|C|} \sum_{T \in \FF_\GGG^C(X_{ij}, \lambda)} W_\GGG(T)\\
&= \frac{1}{|C|} COutside_\GGG^C(X_{ij}) CInside_\GGG^C(X_{ij}, \lambda).
\end{align*}

\end{obs}

In particular, for open curves, $P(C \mid\GGG) = Inside_{\GGG}^{C}(S_{0n})$, and for closed curves, 
$P(C\mid \GGG) = \frac{1}{|C|} \sum_{i=0}^{n-1} CInside_\GGG^C(S_{ii})$.
Thus, for open curves,
$$Count_a(\langle X_{ij} \to \lambda \rangle) = 
\frac{Outside_{\GGG^{(t)}}^{C_a}(X_{ij}) Inside_{\GGG^{(t)}}^{C_a}(X_{ij},\lambda)}{
Inside_{\GGG^{(t)}}^{C_a}(S_{0n})}, $$
and for closed curves,
$$Count_a(\langle X_{ij} \to \lambda \rangle) = 
\frac{COutside_{\GGG^{(t)}}^{C_a}(X_{ij}) CInside_{\GGG^{(t)}}^{C_a}(X_{ij},\lambda)}{
\frac{1}{|C_a|} \sum_{i=0}^{n-1} CInside_{\GGG^{(t)}}^{C_a}(S_{ii})}, $$

To summarize,

\begin{centering}
\framebox{
\parbox{5in}{
\begin{align*}
  Q^{(t)}(\GGG) &= \sum_a \sum_{i,j,k} \sum_{[X\to YZ]\in \RRR}
  Q^{(t)}_{a,i,j,k,[X\to YZ]}(\GGG)\\
&+ \sum_a \sum_{i} \sum_{[X\to \ell]\in \RRR}
  Q^{(t)}_{a,i,[X\to \ell]}(\GGG)\\
&+ \log P(\GGG)
\end{align*}

\begin{align*}
  Q^{(t)}_{a,i,j,k,[X\to YZ]}(\GGG) &= 
\log P_{sub}(\langle X_{ik} \to Y_{ij}
Z_{jk}\rangle ) \cdot   Count_a^{(t)}(\langle X_{ik} \to Y_{ij} Z_{jk} \rangle)\\
Q^{(t)}_{a,i,[X\to \ell]}(\GGG) &=
\log P_{sub}(\langle X_{i i+1} \to \ell_{i i+1}\rangle)
\cdot Count_a^{(t)}(\langle X_{ii+1} \to \ell_{i i+1} \rangle) \\
\end{align*}

\begin{align*}
Count_a(\langle X_{ij} \to \lambda \rangle) &= 
\frac{Outside_{\GGG^{(t)}}^{C_a}(X_{ij}) Inside_{\GGG^{(t)}}^{C_a}(X_{ij},\lambda)}{
Inside_{\GGG^{(t)}}^{C_a}(S_{0n})}, 
\intertext{for open curves, and}
Count_a(\langle X_{ij} \to \lambda \rangle) &= 
\frac{COutside_{\GGG^{(t)}}^{C_a}(X_{ij}) CInside_{\GGG^{(t)}}^{C_a}(X_{ij},\lambda)}{
\frac{1}{|C_a|} \sum_{i=0}^{n-1} CInside_{\GGG^{(t)}}^{C_a}(S_{ii})}, 
\intertext{for closed curves.}
\end{align*}
}
}
\end{centering}

\marginnote{End of learning/doing\_em.tex}
