% sdf2.tex

\subsection{Decomposition Families}

\begin{defn}
  Let $C$ be a curve of length $n$.  A {\em decomposition family for
    $C$} is a pair $(\III, \DDD)$, where $\III$ is a set of intervals
  $[i,j]$, and $\DDD$ is a set of decompositions of the form
$$[i,j] \to [i,k] + [k,j],$$ where $[i,j], [i,k], [k,j]\in \III$. $\III$
must contain the interval $[0,n]$.
\end{defn}

The most trivial decomposition family is the set of all intervals
$[i,j]$ such that $0\le i < j \le n$, together with all possible
decompositions of each $[i,j]$ into $[i,k] + [k,j]$. We will call this
the {\em complete decomposition family}.

\begin{defn}
  A decomposition family $\FFF=(\III,\DDD)$ is $\theta$-flexible if,
  for every interval $[i,j]\in \III$, and every $k$ such that $i<k<j$,
  there is a decomposition
$$[i,j] \to [i,k'] + [k',j],$$
where 
$$|k-k'| < \theta |j-i|.$$
\end{defn}
A $\theta$-flexible decomposition family is desirable because, for any
parse tree, there is a similar parse tree that is allowable under the
decomposition family. Here, similar means that, as we recursively
subdivide a curve into subcurves, we can always choose a midpoint
whose index differs from a given midpoint by at most 10\% of the
length of the subcurve. A $\theta$-flexible decomposition family thus
approximates the complete decomposition family, in some sense.

Another notion of approximation is $\theta$-completeness:
\begin{defn}
  Let $C$ be a curve of length $n$. Let $\FFF=(\III,\DDD)$ be a
  decomposition family for $C$.  An interval $[i,j]$ is called {\em
    reachable} if there is a sequence of intervals $x_i$ such that
  $x_0=[0,n], x_k=[i,j]$, and there is a decomposition with $x_i$ on
  the left-hand side and $x_{i+1}$ on the right-hand side for every
  $i$ between $0$ and $k$.
\end{defn}

\begin{defn}
  A decomposition family $\FFF=(\III,\DDD)$ is $\theta$-complete if,
  for every interval $[i,j], 0\le i< j \le n$, there is a reachable
  interval $[i',j']\in \III$ such that
$$|i-i'| < \theta |j-i|$$
and
$$|j-j'| < \theta |j-i|.$$
\end{defn}

\subsection{Constructing Sparse Decomposition Families for Fast Parsing}

Decomposition families are interesting because they allow us to speed
up parsing by summing or searching over a restricted set of parses:
\begin{thm}
  Let $G$ be a context-free grammar with $k$ rules. Let $C$ be a
  curve of length $n$. Let $\FFF=(\III,\DDD)$ be a decomposition
  family for $C$. Then there is a dynamic programming algorithm to
  approximately (relative to $\FFF$) parse $C$ in time $O(|\DDD|k)$.
\end{thm}
Traditional parsing corresponds to the complete decomposition family,
which has size $O(n^3)$. We can construct sparse decomposition
families of size $O(n)$.

\begin{thm}
  Let $C$ be a curve of length $n$, and let $k$ be an integer. Then
  there is a $\frac{1}{k}$-flexible decomposition family $(\III,\DDD)$
  that has at most $2nk^2$ decompositions.
\end{thm}

\begin{proof}
  We recursively define a sequence of subsampled curves $C^{(i)}$ as
  follows: $C^{(0)} = C$, and $C^{(i+1)}$ is obtained from $C^{(i)}$
  by taking every other point. The number of points in all $C^{(i)}$
  combined is at most $2n$.

  A subcurve between points $p$ and $q$ is \emph{allowable} if there
  is some $C^{(i)}$ which contains both $p$ and $q$, and the subcurve
  between them has length at most $k$ in $C^{(i)}$, i.e., there are at
  most $k-1$ other points between them in $C^{(i)}$. The number of
  allowable subcurves is then at most $2nk$.

  We will say that a subcurve between points $p$ and $q$ \emph{lives}
  in level $i$ (for $i>0$) if $p$ and $q$ are both in $C^{(i)}$, and
  if the length of the subcurve is strictly greater than $k/2$ and at
  most $k$ in $C^{(i)}$. When $i=0$, we will not have the minimum
  length requirement, so all subcurves of $C$ of length at most $k$
  live in level $0$. It is straightforward to see that each curve
  lives in a single level.

  If $p,q$ is a subcurve that lives in level $i$, then we can
  decompose it by picking midpoints from $C^{(i)}$. Some of these
  choices will lead to subcurves that live in level $i-1$.

  This decomposition family is $\frac{1}{k}$-flexible. When picking
  midpoints at level $i$, our subcurve has length at least $k/2$ in
  $C^{(i)}$, so we have at least $k/2$ evenly spaced midpoint
  choices. We can thus choose a midpoint that is within $c$ points of
  any desired midpoint, where $c$ is at most $\frac{1}{k}$ times the
  length of the subcurve we are splitting.
  
  There are at most $k$ midpoints per allowable subcurve, so the number
  of composition rules is at most $2nk^2$.
\end{proof}


% \begin{thm}
% Let $C$ be a curve of length $N = 2^n$. Then there is a
% $\frac{1}{2^{t+1}}$-flexible decomposition family $(\III,\DDD)$ that
% has at most $(2^{3t} + 2^{2t}) N$ decompositions. (So, for
% $\theta=\frac{1}{8}$, $|\DDD|$ is at most $80 N$.)
% \end{thm}

% \begin{proof}
% We represent subcurves of $C$ by intervals of indices of the form
% $[x,y]$, where $0\le x < y \le N$.

% We make the following definitions:
% $$k_{max} = \lfloor \frac{n}{t} \rfloor$$
% $$\LLL_k = \{ [x,y] : 2^{kt} | x, 2^{kt} | y \} \mbox{ for } 0 \le k \le k_{max}.$$ 

% $$\mbox{(Note that } \LLL_0 \supset \LLL_1 \supset \cdots \supset \LLL_{k_{max}}.)$$

% $$\AAA_k = \{ [x,y] \in \LLL_k : (y-x) \le 2^{(k+2)t}\}$$

% $$\NNN_k = \{ [x,y] \in \AAA_k : [x,y] \notin \AAA_j \mbox{ for } j
% < i \}$$

% $$\DDD_k = \{ [x,y] \to [x, z] [z, y] : [x,y] \in
% \NNN_k, z = x + a2^{kt}, x < z < y \}\mbox{ for } 1\le k \le k_{max}$$

% Our decomposition family is $(\NNN, \DDD)$, where
% $\NNN = \cup_{k=0}^{k_{max}} \NNN_k$ and
%  $\DDD = \cup_{k=1}^{k_{max}} \DDD_k$.

% First, we claim that for $[x,y]\to [x,z][z,y]\in \DDD_k$, $[x,z]$ and
% $[z,y]$ are in $\AAA_k$. This holds because:
% $$[x,y]\in \NNN_k \implies 2^{kt} | x,y \implies 2^{kt} | z = x + 2^{kt}a$$
% and
% $$[x,y]\in \NNN_k \implies (y-x) \le 2^{(k+2)t},$$ and since $[x,z],
% [z,y]$ are shorter than $[x,y]$, the same applies to them.

% Second, we claim that every subcurve $[x,y]\in \NNN_k$ has at least
% $2^t - 1$ evenly spaced choices of $z$ such that $([x,y] \to [x,z]
% [z,y]) \in \DDD_k$. This is true if $(y-x) \ge 2^t \cdot 2^{kt} =
% 2^{(k+1)t}$, which is true, because $[x,y] \in \LLL_{k-1}$ and
% $[x,y]\notin \AAA_{k-1}$.

% Since we have at least $2^t - 1$ evenly spaced choices of midpoint, we
% can choose the midpoint within $2^{t+1} (y-x)$ of any desired
% midpoint.

% Now we must bound the size of $\DDD$.
% \begin{align*}
% |\DDD| &= \sum_{k=1}^{k_{max}} |\DDD_k| \\
% &\le \sum_{k=1}^{k_{max}} |\NNN_k| \cdot \frac{\max_{[x,y]\in\NNN_k} y-x }{2^{kt}}\\
% &\le \sum_{k=1}^{k_{max}} |\NNN_k| \cdot 2^{(k+2)t} / 2^{kt}\\
% &\le \sum_{k=1}^{k_{max}} |\AAA_k| 2^{2t}\\
% &\le \sum_{k=1}^{k_{max}} 2^{n - kt} \cdot (2^{2t} - 1) \cdot 2^{2t}\\
% &= 2^{n+2t}(2^{2t}-1) \sum_{k=1}^{k_{max}} 2^{-kt}\\
% &= 2^{n+2t}(2^{2t}-1) \frac{2^{-t}}{1 - 2^{-t}}\\
% &= 2^{n+2t}(2^{t}-1)(2^t + 1) \frac{1}{2^t - 1}\\
% &= 2^{n+2t}(2^t + 1)\\
% &= (2^{3t} + 2^{2t}) N
% \end{align*}

% \end{proof}
