\marginnote{beginning of structure/constituency.tex}


\subsection{Constituency Cues}
In \cite{basri-jacobs, hoffman-richards, Kimia2003Euler}, it is argued that some shapes have natural
constituents, and some shapes do not. Some cues that we believe would
give us some clues to structure: 
\begin{itemize}
\item Large-scale curvature. If we smooth the points of a curve with a
  Gaussian, then we can measure curvature at different levels. We feel
  that large-scale curvature is likely to occur only at constituent
  boundaries.
\item Protuberances. If the curve goes out from a point $x$ and then
  returns to a point $y$, where the length of the curve is
  significantly larger than the distance between $x$ and $y$, then the
  curve from $x$ to $y$ is likely to be a natural constituent. This
  will also be true of any points close to $x$ and $y$, so we should
  restrict ourselves to pairs $(x,y)$ for which the ratio 
$$\frac{\max_{z}||x-z|| + ||y-z||}{||x-y||}$$
is at a local maximum.
\end{itemize}

\marginnote{end of structure/constituency.tex}
