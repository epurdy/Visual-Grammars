% parsing.tex 

\marginnote{beginning of parsing.tex}

We have defined a statistical model that generates curves, and now we
need to know how to calculate the likelihood of a curve under that
model. We do this by parsing curves with PCFSG's.

\subsection{Defining a Parse and Writing the Likelihood}

Let $\GGG = (\NNN, \RRR, \{S\}, \ell, \MMM, \XXX)$ be a shape grammar in
Chomsky Normal Form, and let $C$ be a curve. To simplify matters, we
are assuming that our curve is open and that our grammar has starting
set $\SSS = \{S\}$. Adapting to the general case is straightforward.

Intuitively, we build parses up recursively, starting by parsing line
segments with nonterminals $X$ such that $[X\to \ell]\in
\RRR(X)$. Then, if $[X\to YZ]\in \RRR(X)$, and we have a parse of $D$
starting from the nonterminal $Y$, and a parse of $E$ starting from
$Z$, we can combine these to get a parse of $DE$ starting from $X$.

\begin{defn}
Let $C$ be a curve of length $n$, with points $C[0]$ through $C[n]$.

A $\GGG$-parse of $C$ is a binary tree $T$. Every node of $T$ is
labeled by a tuple $(X,R,i,j,k)$, where $X\in \NNN, R\in \RRR(X)$,
$0\le i < k \le n$, and $j$ is either an integer strictly between $i$
and $k$, or a special symbol $\bot$. The following properties must
hold: 
\begin{itemize}
\item $T$'s root node must be of the form $(S,R,0,j,n)$.
\item If $v$ is an internal node of $T$, then $v$ is of the form $(X,
  [X\to YZ], i, j, k)$, where $i < j < k$. Furthermore, $v$ has
  exactly two children, which are of the form $(Y,R_Y,i,i',j)$ and
  $(Z, R_Z, j, j', k)$.
\item If $v$ is a leaf node of $T$, then $v$ is of the form $(X, [X\to
  \ell], i, \bot, i+1)$.
\end{itemize}
We will denote the set of such parse trees by $Parse_\GGG(C)$. 
\end{defn}
The likelihood of $C$ under $\GGG$ can then be written as
$$P(C\mid \GGG) = \sum_{T\in Parse_\GGG(C)} P(C,T\mid \GGG),$$
where
\begin{align}
\label{joint-likelihood}
P(C,T\mid \GGG) = &\prod_{(X,[X\to YZ],i,j,k)\in T} \rho_X([X\to YZ])
\cdot \mu_{X\to YZ}(C[i], C[j], C[k]) \\
\nonumber
\cdot &\prod_{(X,[X\to \ell],i,\bot,i+1)\in T} \rho_X([X\to \ell])
\end{align}

\def\Vbin{\ema{V_{bin}}}
\def\Vlex{\ema{V_{lex}}}
\newcommand\unirule[1]{\ema{[#1 \to \ell]}}
\newcommand\binrule[3]{\ema{[#1 \to #2 #3]}}
\newcommand\parseset[1]{\ema{Parse_{\GGG}(#1)}}

We will often approximate $P(C\mid \GGG)$ by using the most likely
parse instead of summing over all parses, i.e.
$$P(C\mid\GGG) \approx P_{vit}(C\mid\GGG) := \max_{T\in \parseset{C}}
P(C,T\mid \GGG).$$ This is called the Viterbi approximation, and the
most likely $T$ is called the Viterbi parse.

\subsection{Parsing}
\label{sec-parsing}

Calculating the likelihood of a curve $C$ under a shape grammar $\GGG$
involves either summing over parse trees (to find the exact
likelihood) or maximizing over parse trees (to find the Viterbi
approximation). 

Let $\langle L,v,R\rangle$ denote a binary tree with root $v$, left
subtree $L$, and right subtree $R$. Let $Trees(X,i,k)$ be the set of
parse trees whose root node is of the form $(X,R,i,j,k)$. Intuitively,
$Trees(X,i,k)$ is the set of all the ways that the nonterminal $X$ can
be used to parse the curve $C[i:k]$. We can describe this set
recursively by considering all rules with $X$ on the left-hand side,
and all ways to break the curve $C[i:k]$ into two subcurves:
\begin{align*}
\label{recursive-trees}
Trees(X,i,i+1) &= \begin{cases}
\{ \langle \emptyset, (X, [X\to \ell],i,\bot,i+1), \emptyset\rangle\} &
\mbox{ if } [X \to \ell] \in \RRR(X)\\
\emptyset & \mbox{otherwise}\end{cases}
\end{align*}
\begin{align}
Trees(X,i,k) = \big\{ &\langle T_Y, (X, [X\to YZ], i, j, k), T_Z \rangle \big| [X\to YZ] \in \RRR(X), 
\end{align}\begin{align*}
&i < j < k, T_Y\in Trees(Y,i,j), T_Z\in Trees(Z,j,k)\big\}
\end{align*}
if $k > i+1$, and $\emptyset$ otherwise.

Let $\GGG(X)$ be the grammar $\GGG$ with start symbol $X$ instead of
$S$. Then we can recursively compute the likelihood of a subcurve
under a particular parse tree, by grouping the factors of equation
\eqref{joint-likelihood} into those arising from the root node, those
arising from the left subtree, and those arising from the right
subtree:\footnote{We are abusing notation for the sake of clarity,
  since we are specifying all indices relative to $C$. The indices of
  $C[i:k]$ would not agree with those of $C$ when $i>0$. }
\begin{align*}
P\Big(C[i:k], \Big\langle T_L, (X, [X\to YZ], i,j,k), T_R \Big\rangle \Big| \GGG(X)\Big) = &\rho_X([X\to YZ])  \\
\cdot &\mu_{X\to YZ}(C[i],C[j],C[k]) \\
\cdot &P(C[i:j], T_L \mid \GGG(Y)) \cdot P(C[j:k], T_R \mid \GGG(Z))
\end{align*}
Summing over the set of parse trees (and using the fact that this set
factors according to equation \eqref{recursive-trees}) yields a recursive formula for the overall likelihood.
\begin{align*}
P(C[i:k] | \GGG(X)) = \sum_{T\in Parse_{\GGG(X)}(C[i:k])} & P(C[i:k], T \mid \GGG(X))\\
= \sum_{\substack{[X\to YZ] \in \RRR(X)\\i < j < k\\ T_L\in Trees(Y,i,j)\\ T_R\in Trees(Z,j,k)}} &P\Big(C[i:k], \Big\langle T_L, (X, [X\to YZ], i,j,k), T_R \Big\rangle \Big| \GGG(X)\Big)\\
= \sum_{\substack{[X\to YZ] \in \RRR(X)\\i < j < k}} &\Big[\rho_X([X\to YZ]) \\
&\cdot \mu_{X\to YZ}(C[i],C[j],C[k]) \\
&\cdot P(C[i:j] | \GGG(Y)) P(C[j:k] | \GGG(Z))\Big]
\end{align*}
If we replace summation with maximization, we get a recursive formula
for the Viterbi approximation to the likelihood.
\begin{align*}
P_{vit}(C[i:k] | \GGG(X)) = \max_{T\in Parse_{\GGG(X)}(C[i:k])} & P(C[i:k], T \mid \GGG(X))\\
= \max_{\substack{[X\to YZ] \in \RRR(X)\\i < j < k}} &\Big[\rho_X([X\to YZ]) \\
&\cdot \mu_{X\to YZ}(C[i],C[j],C[k]) \\
&\cdot P_{vit}(C[i:j] | \GGG(Y)) P_{vit}(C[j:k] | \GGG(Z))\Big]
\end{align*}
This recursive formulation makes it clear that we can do either
Viterbi parsing or full parsing via dynamic programming in time
$O(|\RRR|\cdot |C|^3)$. This is much like CKY parsing of strings with
context free grammars.

We have only been dealing with open curves in order to simplify the
discussion, but we are also interested in parsing closed curves. This
can be done efficiently in time $O(|\RRR|\cdot |C|^3)$ via
essentially the same algorithm, where we consider subcurves of $C$
that wrap around the artificially chosen ``end'' of $C$.

\marginnote{end of parsing.tex}
