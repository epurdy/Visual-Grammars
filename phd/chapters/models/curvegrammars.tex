% curvegrammars.tex: explaining model of curve grammars

\marginnote{beginning of models/curvegrammars.tex}

In this section, we define Probabilistic Context-Free Shape Grammars,
a probabilistic model that allows us to generate random curves, parse
curves to find their likelihood, and ultimately learn distributions
over a class of curves.

Analogously to nonterminals in a traditional context-free grammar, we
introduce {\em placed symbols}. A placed symbol is of the form
$X_{p,q}$, where $X$ is a member of a finite alphabet $\NNN$ or the
special symbol $\ell$, and $p,q$ are points. $X_{p,q}$ represents an
oriented curve going from $p$ to $q$. The path between the endpoints
is unspecified.

\marginnote{Recall \spk{10}{gdash} and \spk{10}{gline} from the
  section on L-systems, which represented different kinds of curves
  because they were on the left-hand side of different productions.}
We specify the type $X$ so that different nonterminals can be related
by the grammar. Therefore, $X$ should be thought of as specifying a
particular class of paths between any two endpoints. By itself, $X$ is
an {\em abstract symbol} that can be instantiated as a placed symbol
between any $p$ and $q$.

The special symbol $\ell$ denotes a line segment. This takes the place
of terminals in traditional context-free grammars.

\begin{defn}
A {\em placed curvilinear form} (by analogy with a sentential form) is
a sequence of placed symbols
$$ \alpha^{(0)}_{p_0,p_1} \alpha^{(1)}_{p_1,p_2} \cdots
\alpha^{(n-1)}_{p_{n-1},{p_n}},$$ where $\alpha \in \NNN \cup \{\ell
\}$.  As with curves, $p_0$ will be equal to $p_n$ iff the curvilinear
form is closed, and two closed curvilinear forms will be considered
equivalent if they differ by a cyclic permutation.

An {\em abstract curvilinear forms} is a sequence of abstract symbols
(with no associated geometric information)
$$ \alpha^{(0)} \alpha^{(1)} \cdots \alpha^{(n-1)}.$$ We will specify
whether these are open or closed, since there is no other way to
tell. Again, closed abstract curvilinear forms will be considered
equivalent if they differ by a cyclic permutation.
\end{defn}

We next introduce substitutions and substitution rules, which allow us
to transform curvilinear forms, ultimately producing curves. We will
perform substitutions of the form
$$ X_{p,q} \to Y^{(1)}_{p,p_1} Y^{(2)}_{p_2,p_3} \cdots
Y^{(k)}_{p_{k-1},q}.$$ Since $X_{p,q}$ represents an unspecified path
from $p$ to $q$, substitution simply gives a more specific route, in
terms of which points we will visit in between (the $p_i$, which we
will call the midpoint if there is only one of them, and {\em control
  points} otherwise) and what sort of path we will follow between
these points (the $Y^{(i)}$).

In order to give a substitution rule for performing substitutions, we
need to give
\bitem
\item An {\em abstract substitution rule} $X\to Y^{(1)}\cdots Y^{(k)}$.
\item a rule for determining the $p_i$ in terms of $p$ and $q$.
\eitem

In practice, applying a substitution rule is problematic because the
$p_i$ live in an infinite domain ($\RR^2$), but we want to deal with
curves that (1) live in a finite domain (the pixels of the image
plane) and (2) are expected to exhibit a good deal of variation. Thus,
we give a distribution $\mu_{X\to Y^{(1)}\cdots Y^{(k)}}(p_1,\dots,
p_{k-1} ; p,q)$ over the $p_i$ called the {\em control point distribution}. 
When there is only a single control point, we will call $\mu_{X\to
YZ}(p_1; p,q)$ the {\em midpoint distribution}.

\begin{defn}
A probabilistic context-free shape grammar (PCFSG) is a tuple 
$\GGG = (\NNN, \RRR, \SSS, \ell, \MMM, \XXX)$, where
\bitem
\item $\NNN$ is a set of abstract symbol types, which we will call
  {\em nonterminals}
\item $\RRR$ is a set of abstract substitution rules, with $\RRR(X)$
  being the set of rules in $\RRR$ with $X$ on the left-hand side
\item $\SSS$ is the {\em starting set} of abstract curvilinear forms
\item $\ell$ is a special curve type representing a line segment
\item $\XXX = \{ \rho_X \mid X\in \NNN \} \cup
  \{\rho_\SSS\}$, where $\rho_X$ is a probability distribution over
  $\RRR(X)$, and $\rho_\SSS$ is a distribution over $\SSS$
\item $\MMM = \{ \mu_{X\to Y^{(1)}\cdots Y^{(k)}} \mid (X\to
  Y^{(1)}\cdots Y^{(k)}) \in \RRR\}$ is a set of control-point
  distributions.
\eitem
\end{defn}

It is worth noting that $\GGG$ has a corresponding {\em abstract
  grammar} $\GGG_{abs} = (\NNN, \RRR, \SSS, \{\ell\}, \XXX)$ which is
just a traditional PCFG. $\GGG_{abs}$ is an odd PCFG because it only
generates strings of the symbol $\ell$; nevertheless, many properties
of $\GGG$ are actually properties of $\GGG_{abs}$.

For a grammar that generates open curves, we can assume that $\SSS$
has a single curvilinear form $S$, and we will call $S$ the start
symbol. We sample from such a PCFSG by starting with the curvilinear
form $S_{p,q}$ for arbitrary $p$ and $q$. While our curvilinear form
contains a placed symbol $X_{p',q'}$ such that $X\ne \ell$, we pick a
random substitution rule $X \to Y^{(1)} \dots Y^{(k)}$ according to
$\rho_X$, and pick random control points $p_1,\dots, p_{k-1}$
according to $\mu_{X\to Y^{(1)}\dots Y^{(k)}}(p_1, \dots, p_{k-1}; p',
q')$. We then replace the placed symbol $X_{p',q'}$ with the
curvilinear form $Y_{p',p_1}^{(1)} \dots Y_{p_{k-1},q'}^{(k)}$. We
will disallow substitutions in which any two control points $p_i, p_j$
are equal, or in which any control point $p_i$ is equal to either of
the endpoints $p'$ or $q'$.

\marginnote{Explain this better}
There is a slight difficulty here in that we usually cannot define the
control-point distribution if $p$ and $q$ are equal. This is
important, since we are mainly interested in closed curves, so we
would like to start with $S_{p,p}$ for some arbitrary $p$. In this
case, our set $\SSS$ of starting forms will contain abstract
curvilinear forms of length two or more, which are understood to be
closed. We choose a random such curvilinear form according to
$\rho_\SSS$, and then continue as before.

\subsection{Restricted Classes of Shape Grammars}

In this section, we define some restricted classes of PCFSG's. Our
restrictions are only on the abstract part of the grammar, so we are
just specifying restricted classes of traditional grammars.

For simplicity and efficiency, we will usually restrict our abstract
grammars to be in Chomsky Normal Form, in which abstract substitution
rules can only be of two forms: 
\bitem
\item Binary rules of the form $X \to Y Z$.
\item Lexical rules of the form $X \to \ell$.
\eitem 

From now on, we will assume that PCFSG's are in Chomsky Normal Form.
Accordingly, we will speak of midpoints, rather than control
points. It is important to note that PCFSG's differ from standard
PCFG's in that not all PCFSG's are equivalent to a PCFSG in Chomsky
Normal Form. This is because not all control point distributions can
be realized as a product of midpoint distributions. For instance,
suppose we have a rule $A\to BCD$, which in the PCFG setting could be
replaced with rules $A\to BX, X\to CD$. We could have a control point
distribution $\mu_{A\to BCD}(p_1,p_2;p,q)$ in which $p_1,p_2$ were on
a random circle going through $p$ and $q$. If we try to replicate such
a control point distribution with midpoint distributions $\mu_{A\to
  BX}(p_1; p,q), \mu_{X\to CD}(p_2; p_1, q)$, we cannot, because the
distribution $\mu_{X\to CD}(p_2; p_1,q)$ can only depend on the two
points $p_1$ and $q$, and that is not sufficient to specify a unique
circle.

\marginnote{end of models/curvegrammars.tex}
