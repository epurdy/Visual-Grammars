
\subsection{Building a Grammar for a Single Curve}
\label{sec-single}

The current work is based on \cite{hcm}, in which grammar-like
structures were constructed from single training curves. We follow
that approach here.

A grammar $\GGG_C$ for a curve $C$ should produce $C$, and curves
which look like $C$. We are not trying to achieve structural variation
yet, so the only variation that $\GGG_C$ needs to account for is:
\begin{itemize}
\item Slight geometric deformations of $C$
\item Curves like $C$ that have more or fewer sample points.
\end{itemize}
We model geometric deformations with the nonparametric distribution
described in Section \ref{sec-deform}. We model curves with fewer sample
points by allowing every nonterminal $X$ to have the rule $[X\to
\ell]$. We initially set the probability of this rule based on the
scale of $X$, as described in Section \ref{sec-scale}.

The nonterminals of our grammar will correspond to subcurves of $C$.
We model curves with more sample points than $C$ by allowing any
nonterminal $X$ that arises from a subcurve of length $1$ to have two
rules:
\begin{itemize}
\item $X\to \ell$
\item $X\to XX$
\end{itemize}
The second rule is infinitely recursive, and thus allows the curve to
get arbitrarily long.
  \marginnote{Our current thinking is to have the
  probability of $X\to XX$ be chosen in the same scale-dependent way
  as the probability of $X\to YZ$. But this might be problematic.}

\subsection{General Case}

Let $C$ be a curve of length $n$. A \emph{generic parse tree} for $C$
is a parse tree that does not refer to any particular grammar. This
will just be a binary tree $T$ where each node is labeled by an
interval $(i,j)$, $0\le i < j \le n$, and
\begin{itemize}
\item The root of $T$ is $(0,n)$
\item If $(i,j)$ is a node, and $j-i \ge 2$, then $(i,j)$ has exactly
  two children, and they are of the form $(i,k)$ and $(k,j)$ for some
  $k$.
\item $T$ has $n$ leaf nodes $(i,i+1)$ for $i=0,\dots,n-1$.
\end{itemize}

We can then specify a simple grammar for $C$ that produces the parse
tree $T$. For every node $(i,j)$ in $T$, $\GGG$ has a nonterminal
$X^{(i,j)}$. When $(i,j)$ has children $(i,k)$ and $(k,j)$, $\GGG$ has
a rule $[X^{(i,j)} \to X^{(i,k)}X^{(k,j)}]$. The scale of the
nonterminal $X^{(i,j)}$ is $\frac{j-i}{n}$ (see Section \ref{sec-scale}).

Our choice of an initial grammar for $C$ is thus based on choosing a
generic parse for $C$. Some choices:
\begin{enumerate}
\item An arbitrary parse, chosen to make the tree as balanced as
  possible. This approach corresponds most closely to that of
  \cite{hcm}.
\item A parse that has been chosen to respect the structure of
  $C$. This approach would require us to identify natural constituents
  of curves. There has been some work on this question.
\item We can choose multiple parses, and create a grammar that can
  parse $C$ in multiple ways. 
\end{enumerate}

Our favored approach is number 3. We want to create a small grammar
that can parse $C$ in as many ways as possible, so that the EM
retraining can discover the ``correct'' sub-grammar. To this end, we
use the sparse decomposition families of Section
\ref{sec-fast-parsing}.




