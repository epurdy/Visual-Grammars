\documentclass{article}
\usepackage{leonine,amsmath,amssymb,amsthm,graphicx, setspace}%%xy, setspace, amscd (commutative diagram)
\title{Linear Time Parsing with Sparse Decomposition Families}
\author{Pedro Felzenszwalb \footnote{Department of Computer Science, University of Chicago. Email: pff@uchicago.edu}\\
Eric Purdy \footnote{Department of Computer Science, University of Chicago. Email: epurdy@uchicago.edu}}

%%\doublespace

\begin{document}
\maketitle

\section{Introduction}
We are interested in modeling time series data. We will consider
sequences of data $C=c_0, \dots, c_n$. We would like to create a
meaningful similarity measure, so that we can do nearest neighbor
classification of sequences. 

\section{Dynamic Time Warping}

One well-known and effective approach to time-series matching is to
use Dynamic Time Warping (DTW), which searches for a correspondence
between the points of two time series such that their data are
similar, and such that the correspondence is relatively smooth.

Dynamic time warping is defined in terms of a distance function $d$
between data points. Let $X$ and $Y$ be two sequences, of length $m$
and $n$ respectively. Then we let the DTW match cost be
$M[m,n]$, where $M$ is defined via the following dynamic program:
\begin{align*}
M[0,0] &= d(x_0, y_0) \\
M[i,j] &= d(x_i,y_j) + \min \{ M[i-1,j], M[i,j-1], M[i-1,j-1]\}.
\end{align*}

It should be noted that this penalizes stretching, since choosing
$M[i-1,j]$ or $M[i,j-1]$ instead of $M[i,j]$ increases the number of
  $d(\cdot,\cdot)$ terms in the final sum.

One weakness of DTW is that it does not model long-range
properties. 

\begin{ex}
\label{stretch-problem}
For example, DTW will charge similar amounts for skipping
$k$ points in a row, and for skipping $k$ points that are spaced far
apart. The latter case is less likely to indicate error, since it can
arise from small variations in the sampling rate. If we consider three
times series $X,Y,Z$:
\begin{align*}
x_i &= i, &i=0, \dots, 2n\\
y_i &= 2i, &i=0, \dots, n\\
z_i &= \begin{cases} 0 & i=0,\dots, n\\ i-n & i=n+1, \dots, 3n\end{cases}\\
\end{align*}
Arguably, $X$ and $Y$ are the same time series at different
resolutions, but they are distance $n$ apart, while $X$ and $Z$ have
different shapes, but are distance $0$ apart.
\end{ex}

\section{Parsing Sequences to Measure Distance}

Instead of growing our sequences by one each time, as in DTW, we want
to consider hierarchical decompositions of the sequences. Let $X$ be a
sequence of length $n$. We consider intervals $[i,j]$ for all $0\le i
\le j \le n$, and decompositions of the form $[i,k] \to [i,j] +
    [j,k]$.  Recall that we have some similarity measure
    $d(\cdot,\cdot)$ on the domain of $X$ and $Y$.

Suppose we have two sequences, $X$ and $Y$, of length $m$ and $n$
respectively, and we wish to measure their similarity. A {\em
  hierarchical matching} is a set $\MMM$ of pairs of intervals $I,J$
such that $([0,m], [0,n])\in \MMM$, and for every $(I,J) \in \MMM$
where $|I|, |J|> 1$, there are two pairs $(I_1, J_1)$, $(I_2, J_2)\in
\MMM$ with $I=I_1 + I_2$ and $J=J_1 + J_2$.

The cost of a matching is the sum of costs for matching splits,
$W_{split}(I\to I_1 + I_2, J\to J_1 + J_2)$, where $(I,J), (I_1,J_1),
(I_2,J_2)\in \MMM$, and costs for matching a given pair of intervals
$W_{pair}(I,J)$, where $(I,J)\in \MMM$.
We can calculate these costs via the following dynamic program:

\begin{align*}
M[[i,i+1],J] &= W_{pair}([i, i+1],J)\\
M[I,[j, j+1]] &= W_{pair}(I,[j,j+1])\\
M[I,J] &= \min_{\substack{I\to I_1 + I_2 \\ J\to J_1 + J_2}} 
M[I_1, J_1] + M[I_2, J_2] + W_{split}(I\to I_1 + I_2, J\to J_1 + J_2) + W_{pair}(I,J)\\
\end{align*}

We have many choices for these cost functions.

\subsubsection{Emulating DTW}
\begin{align*}
W_{pair}([i,i+1], [j_1,j_2]) &= \frac{1}{2}\left(d(x_i,y_{j_1}) +
d(x_{i+1}, y_{j_2})\right) + \min_{j_1 \le k < j_2} \sum_{j=j_1 + 1}^k
d(x_i, y_j) + \sum_{j=k+1}^{j_2 - 1} d(x_{i+1}, y_j)\\
W_{pair}([i_1,i_2], [j,j+1]) &= \frac{1}{2}\left(d(x_{i_1},y_j) +
d(x_{i_2}, y_{j+1})\right) + \min_{i_1 \le k < i_2} \sum_{i=i_1 + 1}^k
d(x_i, y_j) + \sum_{i=k+1}^{i_2 - 1} d(x_{i}, y_{j+1})\\
W_{pair}(I, J) &= 0 \mbox{\quad if } |I|,|J| > 1 \\
W_{split}(I\to I_1 + I_2, J\to J_1 + J_2) &= 0
\end{align*}
This corresponds almost exactly to dynamic time warping - the cost is
the same as the corresponding warp in DTW, but not every warp can be
represented by a matching. For example, matching all but the last
point in $X$ to the first point of $Y$, and matching all but the first
point of $Y$ to the last point of $X$. Reasonable warpings will be
very close to a warping that is representable by a matching.

Note that, like DTW, this penalizes stretching and contracting, because we end up
adding more distance terms together when we stretch.

\subsubsection{Explicit stretching penalty}
We can use
\begin{align*}
W_{split}(I\to I_1 + I_2, J\to J_1 + J_2) &= w(I,J) \times \left[\left( \frac{|I_1|}{|I|} - \frac{|J_1|}{|J|}\right)^2 +
\left( \frac{|I_2|}{|I|} - \frac{|J_2|}{|J|}\right)^2 \right].
\end{align*}
This penalizes matching two intervals which are split in very
different places. It would impose a hefty penalty on matching $X$ and
$Z$ in example \ref{stretch-problem}, since the problematic match
there would be charged $W_{split}([0,2n]\to [0,0] + [0,2n], [0,3n] \to
[0,n] + [n,3n])$.

We would have to use some weighting function $w(I,J)$ to make the
longer intervals count more. For very short intervals, sampling
artifacts would make this very noisy. For example, matching an
interval of length two to an interval of length three would have a
minimum penalty of $\frac{1}{18}$.

\subsubsection{DTW without implicit stretching penalty}
The formula isn't quite right, but it's close.

If we don't want to implicitly penalize stretching, we can use
\begin{align*}
W_{pair}([i,i+1], J) &= \min_{J = J_1 + J_2} \left(1+ \frac{1}{|J_1|}\right) \sum_{j\in J_1} d(x_i,
y_j) + \left(1+ \frac{1}{|J_2|}\right) \sum_{j\in J_2} d(x_{i+1}, y_j)\\
W_{pair}(\{i\}, J) &= \left(1 + \frac{1}{|J|}\right) \sum_{j\in J} d(x_i, y_j) \\
W_{pair}(I, \{j\}) &= \left(1 + \frac{1}{|I|}\right) \sum_{i\in I} d(x_i, y_j) \\
W_{pair}(I, J) &= 0 \mbox{\quad if } |I|,|J| > 1 \\
W_{split}(I\to I_1 + I_2, J\to J_1 + J_2) &= 0.
\end{align*}
It can be verified that each $i$ and $j$ then contributes a weighted
average of one data term, for $m+n$ total data terms.

Since stretching is a bad thing, this seems silly. However, we can
combine this with the explicit stretching penalty.

\subsubsection{Comparing subsequences by approximation}

We can define the cost $W_{pair}(I,J)$ in terms of how similar the
sequences $X$ and $Y$ appear on the respective intervals $I$ and $J$,
by comparing approximations of them. For example, if we had a linear
approximation of each, we could compare the two lines. It should be
possible to compute linear approximations of all subsequences in some
efficient way.

Wavelets might provide a cool way to do this, since we could probably
compare the wavelet coefficients in a semi-principled way.

\subsubsection{Bias and Gain Invariance}

What is base case?

We can be invariant to bias and gain by charging some function of
%% \begin{align*}
%% W_{pair}(\{i\}, J) &= \left(1 + \frac{1}{|J|}\right) \sum_{j\in J} d(x_i, y_j) \\
%% W_{pair}(I, \{j\}) &= \left(1 + \frac{1}{|I|}\right) \sum_{i\in I} d(x_i, y_j) \\
%% W_{pair}(I, J) &= 0 \mbox{\quad if } |I|,|J| > 1 \\
%% W_{split}(I\to I_1 + I_2, J\to J_1 + J_2) &= 0.
%% \end{align*}
$$\frac{C[i_2] - C[i_1]}{C[i_3] - C[i_1]} - \frac{D[i_2] - D[i_1]}{D[i_3] - D[i_1]}.$$
This makes the cost invariant to local bias and gain, which might be a little bit strong.

\subsubsection{Detrending data cost}
We can also formulate a detrending data term, so that if $C'$ is
defined in terms of $C$ as $c'_i = c_i + a + bi$, the cost for
matching $C'$ and $C$ is zero.

To match $[i_1,i_3]\to [i_1,i_2] + [i_2, i_3]$ with $[j_1,j_3]\to [j_1,j_2] + [j_2, j_3]$, we charge
\begin{align*}
&\left[ C[i_2] - C[i_1] + \frac{i_2 - i_1}{i_3 - i_1}(C[i_3] - C[i_1])\right] \\
-&\left[ D[j_2] - D[j_1] + \frac{j_2 - j_1}{j_3 - j_1}(D[j_3] - D[j_1])\right] 
\end{align*}
We can verify that works for detrending: suppose $D[i] = C[i] + a + bi$. Then the cost of matching 
$[i_1,i_3]\to [i_1,i_2] + [i_2, i_3]$ with itself is
\begin{align*}
&\left[ C[i_2] - C[i_1] + \frac{i_2 - i_1}{i_3 - i_1}(C[i_3] - C[i_1])\right] \\
-&\left[ C[i_2] + a + bi_2 - C[i_1] -a - bi_1 + \frac{i_2 - i_1}{i_3 - i_1}(C[i_3] + a + bi_3 - D[j_1] - a - bi_1)\right],
\end{align*}
which is zero. The cost is also approximately zero if $D$
simultaneously includes an additive trend and is stretched, although
small errors will result from subsampling. 

The correct base case is probably to compute the cost of matching each
interval $J$ to an interval $[i,i+1]$; this will not depend on which
$i$ we pick, and will only measure how far $D$ or $C$ deviates from a
straight line in the given interval.


\section{Decomposition Families}

\begin{defn}
Let $C=c_0,\dots,c_n$ be a sequence. We will say that $C$ has length
$n$, even though it has $n+1$ points.

A {\em decomposition family for $C$} is a pair $(I, R)$, where $I$ is
a set of intervals $[i,j]$, and $R$ is a set of rules of the form
$$[i,j] \to [i,k] + [k,j],$$ where $[i,j], [i,k], [k,j]\in I$. $I$
  must contain the interval $[0,n]$.
\end{defn}

The most trivial decomposition family is the set of all intervals
$[i,j]$ such that $0\le i < j \le n$, together with all possible
decompositions of each $[i,j]$ into $[i,k] + [k,j]$. We will call this
the {\em complete decomposition family}.

Decomposition families are interesting because they allow us to speed
up parsing by summing or searching over a restricted set of parses:
\begin{thm}
  Let $G$ be a context-free grammar with $k$ rules. Let $C$ be a
  sequence of length $n$. Let $\FFF=(\III,\DDD)$ be a decomposition
  family for $C$. Then there is a dynamic programming algorithm to
  approximately (relative to $\FFF$) parse $C$ in time $O(|\DDD|k)$.
\end{thm}
Traditional parsing corresponds to the complete decomposition family,
which has size $O(n^3)$. We can construct sparse decomposition
families of size $O(n)$, though.

\begin{defn}
A decomposition family $F=(I,R)$ is $\theta$-flexible if, for every interval
$[i,j]\in I$, and every $k$ such that $i<k<j$, there is a rule
$$[i,j] \to [i,k'] + [k',j],$$
where 
$$|k-k'| < \theta |j-i|.$$
\end{defn}

\begin{defn}
Let $C$ be a sequence of length $n$ Let $F=(I,R)$ be a decomposition
family for $C$.  An interval $[i,j]$ is called {\em reachable} if
there is a sequence of intervals $x_i$ such that $x_0=[0,n],
x_k=[i,j]$, and there is a rule with $x_i$ on the left-hand side and
$x_{i+1}$ on the right-hand side for every $i$ between $0$ and $k$.
\end{defn}

\begin{defn}
A decomposition family $F=(I,R)$ is $\theta$-complete if, for every
interval $[i,j], 0\le i< j \le n$, there is a reachable interval
$[i',j']\in I$ such that
$$|i-i'| < \theta |j-i|$$
and
$$|j-j'| < \theta |j-i|.$$
\end{defn}

\section{Constructing Sparse Decomposition Families}

\begin{thm}
  Let $C$ be a sequence of length $n$, and let $k$ be an integer. Then
  there is a $\frac{1}{k}$-flexible decomposition family $(\III,\DDD)$
  that has at most $2nk^2$ decompositions.
\end{thm}

\begin{proof}
  We recursively define a sequence of subsampled sequences $C^{(i)}$ as
  follows: $C^{(0)} = C$, and $C^{(i+1)}$ is obtained from $C^{(i)}$
  by taking every other point. The number of points in all $C^{(i)}$
  combined is at most $2n$.

  A subsequence between points $p$ and $q$ is \emph{allowable} if
  there is some $C^{(i)}$ which contains both $p$ and $q$, and the
  subsequence between them has length at most $k$ in $C^{(i)}$, i.e.,
  there are at most $k-1$ other points between them in $C^{(i)}$. The
  number of allowable subsequences is then at most $2nk$.

  We will say that a subsequence between points $p$ and $q$
  \emph{lives} in level $i$ (for $i>0$) if $p$ and $q$ are both in
  $C^{(i)}$, and if the length of the subsequence is strictly greater
  than $k/2$ and at most $k$ in $C^{(i)}$. When $i=0$, we will not
  have the minimum length requirement, so all subsequence of $C$ of
  length at most $k$ live in level $0$. It is straightforward to see
  that each subsequence lives in a single level.

  If $p,q$ is a subsequence that lives in level $i$, then we can
  decompose it by picking midpoints from $C^{(i)}$. Some of these
  choices will lead to subsequences that live in level $i-1$.

  This decomposition family is $\frac{1}{k}$-flexible. When picking
  midpoints at level $i$, our subsequence has length at least $k/2$ in
  $C^{(i)}$, so we have at least $k/2$ evenly spaced midpoint
  choices. We can thus choose a midpoint that is within $c$ points of
  any desired midpoint, where $c$ is at most $\frac{1}{k}$ times the
  length of the subsequence we are splitting.
  
  There are at most $k$ midpoints per allowable subsequence, so the number
  of composition rules is at most $2nk^2$.
\end{proof}


\section{Further Questions}

\subsection{What sorts of grammars are robust to the approximation?}
In some cases, we may need to modify a grammar slightly so that it
behaves well with approximate parsing.

For instance, consider approximate parsing of strings with space
removed. If we use a context-free grammar trained on English, the
approximation would be very, very inaccurate, since being a single
letter off will prevent us from recognizing a word as being a
particular part of speech.

In this case, we could modify our rules so that
\begin{align*}
N &\to \mbox{cat} \\
\intertext{becomes}
N &\to \mbox{cat} \\
N &\to \mbox{at} \\
N &\to \mbox{cath} \\
N &\to \mbox{cata} \\
N &\to \mbox{catg} \\
\ldots
\end{align*}
That is, we allow letters to be removed at either end, and allow
letters to be added at either end. When adding letters, the
distribution should be realistic (such as from an n-gram model) rather
than uniform, or else we will pay $\log|\Sigma|$ for every letter
added.

In general, we can imagine adapting any grammar to deal with such
problems. The grammar must allow deletion and addition of data at the
ends of a chunk. The model for addition should be relatively accurate.

\subsection{Data-Dependent Decomposition Families}

Basically, want to build families on top of a good subsampling. Here
good means that it respects some property of the sequence, which could
mean either boundary cues or that the subsampled version approximates
the original well.

\subsection{Correcting Parses through Local Perturbations}

We think it is meaningful to talk about local perturbations of a
parse. This would hopefully improve the quality of the parse.

We are assuming that all grammars are in CNF, and thus all parse trees
are binary trees.

Consider every pair of adjacent data points. Some pairs are not
separated by a constituency boundary (if the two points were the two
children of an interval of length two), but at least half of the pairs
are. 

It is then reasonable to consider shifting either of these data points
across the constituency boundary. We have to somehow rearrange the
parse tree to keep it binary. 

We want to have a local computation at each of these adjacent
pairs. How do we do that?

There are a linear number of adjacent pairs. How quickly can we do all
the perturbations we want to do?

\subsection{Approximation guarantees}
Can we prove bounds on the cost incurred by moving to sparse
decomposition families? For example, if we use cost functions
emulating DTW, what is the cost incurred by using SDF's instead of
complete decomposition family? As far as I can tell, no such bound is
obtainable. The conditions I investigated were when the distance
function and the sequences satisfied
$$|d(x_i, y_j) - d(x_k, y_\ell) \le C |i-k| + |j-\ell|.$$ It seems
that we can then figure out how much cost is incurred by applying
Green's theorem in the DTW square, and the cost is too high. That is,
it can be larger in magnitude than ...


\end{document}
