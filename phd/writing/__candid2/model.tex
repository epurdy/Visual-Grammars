\documentclass{article}
\usepackage{leonine,amsmath,amssymb,amsthm,graphicx}%%xy, setspace, amscd (commutative diagram)
\title{Notes}
\author{Eric Purdy \footnote{Department of Computer Science, University of Chicago. Email: epurdy@uchicago.edu}}

%%\doublespace

\begin{document}
\maketitle

\section{Rules of Thumb}
\begin{itemize}
\item Glutamate is black ink of computation
\item GABA is red ink of computation
\item Apical dendrites are the input dendrites
\item Basal dendrites are the output dendrites
\end{itemize}

\section{Cerebellum}

\subsection{Cell Types}

\begin{description}
\item[Source Neuron (Pontine Nuclei)] Axons are called ``Mossy
  Fibers''. There are forty times more Source Neurons than Sink
  Neurons.[PONS 831]
\item[Sink Neuron (Deep Cerebellar Nuclei, Lateral/Medial Vestibular
  Nuclei of the Brain Stem)]  
\item[Production Neuron (Granular cell)] Axons are called ``parallel
  fibers''. There are $10^{11}$ such cells, approximately half of the
  cells in the brain. [PONS 835]
\item[Right Anti-Symbol Neuron (Purkinje)] Output is entirely
  inhibitory and mediated by GABA (red ink). [PONS 835]
\item[Right Exception Neuron (Inferior Olive)]
\item[Golgi cell]
Input dendrites in molecular layer
GABA (red ink) terminals output (axodendritically) in the glomeruli

Looks like an anti-production. So some productions also generate anti-productions?
\item[Basket Cell]
\item[Stellate Cell]
These two cell types are inhibiting the Purkinjes, so they are
ultimately giving increased weight to various symbols.
\end{description}

\subsection{Cerebellar Glomeruli}

Pontine nuclei terminals contact granule cells and Golgi neurons in
synaptic complexes called CG.

\subsection{Symbolic Coherence and Equivalence}

We hypothesize that neurons which carry information over significant
distances are encoding some comprehensible meaning. We will thus refer
to certain neurons as conveying symbols. We will also say that groups
of related symbols exist in symbol-clusters.

\begin{conj}[Symbolic Equivalence Hypothesis] Symbols in one part of
  the brain can be paired up with equivalent or roughly equivalent
  symbols in another part of the brain.
\end{conj}

\begin{conj}[Symbolic Coherence Hypothesis] Symbols that are related
  in one part of the brain will be anatomically related.
\end{conj}


[PONS 837]:
\begin{quote}
  Individual Purkinje neurons receive synaptic input from only a
  single climbing fiber, whereas each climbing fiber contacts 1-10
  Purkinje neurons. The terminals of the climbing fibers in the
  cerebellar cortex are arranged topographically; the axons of
  clusters of olivary neurons terminate in thin parasagittal strips
  that extend across several folia. In turn, the Purkinje neurons
  within each strip project to common groups of deep nuclear
  neurons. This highly specific connectivity of the climbing fiber
  system contrasts markedly with the massive convergence and
  divergence of the mossy and parallel fibers.
\end{quote}

\begin{quote}
Individual Anti-Symbol neurons receive exception signals from only a
single Exception neuron. Each Exception neuron contacts 1-10
Anti-Symbol neurons (presumably a coherent symbol-cluster). The
terminals of the Exception neurons are arranged topographically;
clusters of Exception neurons output into thin strips that extend
across several folia. In turn, the Anti-Symbol neurons within each
strip project to common groups of Sink neurons. 
\end{quote}

[PONS 842]:
\begin{quote}
  More refined mapping studies of the cerebellar cortex based on
  single-cell recordings reveal that the input from a given peripheral
  site, such as a local area of skin, diverges to multiple discrete
  patches of granule cells, an arrangement called fractured
  somatotopy.

Recent anatomical studies of primates show that the deep cerebellar
nuclei are also organized somatotopically. They are arranged to
receive projections from the two maps on the dorsal and ventral
surfaces of the intermediate and lateral zones of the cerebellar
cortex and project to the magnocellular red nucleus and primary motor
cortex via the thalamus.
\end{quote}

\begin{quote}
  Symbols (signaled by Source neurons) from a particular cluster
  diverge to multiple discrete patches of Production neurons.

  The Sink neurons also exhibit symbolic coherency. They project to
  the cerebral cortex via the thalamus.

...

\end{quote}

\begin{quote}
  Purkinje neurons in the lateral cerebellar cortex project to the
  dentate nucleus. Most dentate axons exit the cerebellum via the
  superior cerebellar peduncle and have two main terminations. One
  termination is in the contralateral ventrolateral thalamus, in the
  same area receiving input from the interposed nucleus. These
  thalamic cells project to premotor and primary motor areas of the
  cerebral cortex. The second main termination of dentate neurons is
  in the contralateral parvocellular red nucleus, a portion of the red
  nucleus that is distinct from the part receiving input from the
  interposed nucleus. These neurons project to the inferior olivary
  nucleus, which in turn projects back to the contralateral cerebellum
  in the climbing fibers, thus forming a feedback loop. In addition to
  receiving input from the dentate nucleus, parvocellular neurons also
  receive input from the lateral premotor areas. The intriguing
  suggestion has been made, based on brain imaging, that this
  premotor-cerebello-rubrocerebellar loop is involved in the mental
  rehearsal of movements and perhaps with motor learning.
\end{quote}

\subsection{The Pass-Through Pathway}

$$X \to X$$

\begin{itemize}
\item The $X$ Source Neuron sends a positive signal to the $X$ Sink Neuron.
\end{itemize}

\subsection{The Anti-Production Pathway}

$$ X \to \bar{Y} \bar{Z} $$

\begin{itemize}
\item The $X$ Source Neuron sends a positive signal to the $X\to YZ$
  Production Neuron.
\item The $X\to YZ$ Production Neuron sends a positive signal to
  the Right Anti-Symbol Neurons for $Y$ and $Z$.
\item The $Y$ Right Anti-Symbol Neuron sends a negative signal to the
  $Y$ Sink Neuron.
\item The $Y$ Sink Neuron does not fire.
\end{itemize}

\subsection{The Exception Pathway}

$$ X\to \bar{Y} \bar{Z}$$
$$ \bar{Y} $$

\begin{itemize}
\item The $X$ Source Neuron sends a positive signal to the $X\to YZ$
  Production Neuron.
\item The $X\to YZ$ Production Neuron sends a positive signal to the
  Right Anti-Symbol Neuron for $Y$.
\item The $Y$ Right Exception Neuron sends a positive signal to Right
  Anti-Symbol Neuron for $Y$.
\item The $Y$ Right Anti-Symbol Neuron is switched off (the ``complex
  spike'' depolarization).
\item The $Y$ Sink Neuron can start firing. 
\item The link between the $X\to YZ$ Production Neuron and the $Y$
  Right Anti-Symbol Neuron is weakened.
\end{itemize}

The pathway from The Right Exception Neuron to the Right Anti-Symbol
Neuron to the Sink Neuron is said to be an unusually straightforward
connection.


\end{document}