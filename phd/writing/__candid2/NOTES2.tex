\documentclass{article}
\usepackage{leonine,amsmath,amssymb,amsthm,graphicx}%%xy, setspace, amscd (commutative diagram)
\title{Notes}
\author{Eric Purdy \footnote{Department of Computer Science, University of Chicago. Email: epurdy@uchicago.edu}}

%%\doublespace

\begin{document}
\maketitle

\part{Readings}

\cite{centaur} refers to about 8 papers on constituency cues

\part{SPECULATIONS}

\subsubsection{KL Divergence corresponds to Likelihood change(?)}

In replacement, we say that every previous parse with $Y$ is now
  a parse with $X$. We have to pay some cost for this, since
  $P(d_Y|X)$ is probably lower than $P(d_Y|Y)$. The KL divergence
  tells you this cost, sort of.

\subsubsection{Rewarding Simplification(?)}

Replacement simplifies the grammar, since we are deleting all rules
with $Y$ on the left-hand side. It is also possible that some
nonterminals will become inaccessible after the replacement. These
nonterminals are no longer functionally part of the grammar, so we
should delete them and increase the prior probability of the grammar
accordingly.

\note{
If we do step by step modifications, we can probably maintain a data
structure that tells us what things become inaccesible. It's just
accesibility from the start symbol in the (non-hyper)graph of the
grammar.

We initialize the data structure such that every node knows its
incoming edges, and has one of them marked as being on a path from the
root. When we delete a directed edge, the target node may become
inaccessible. If the edge we deleted was not the rootward edge of the
target, we do nothing. If it was, we check all the incoming edges of
the target; if one of them comes from an accessible node, we switch
the targets rootward edge to that edge. If none of them are, we delete
the node and all its edges. 

Need to check the correctness of this algorithm.
}


\part{DISCARDED METHODS AND IDEAS}

\end{document}
