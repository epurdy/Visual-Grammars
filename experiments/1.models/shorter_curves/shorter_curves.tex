
\FloatBarrier

\section{Recover a correspondence where some points are missing}

Here we build a grammar from a ground-truth Romer curve, and try to
parse one of the (much shorter) hand-annotated Romer curves. We can
safely assume that every point in the parsed curve has a corresponding
one in the example curve, which is the reverse of the previous
experiments.

In order to do this successfully, the grammar needs shortening rules,
but not lengthening rules.

\begin{figure}
\includegraphics[width=\linewidth]{experiments/2.parsing/shorter_curves/output.d/parse_00.png}
\caption{On the left, the model curve. On the right, the parsed curve}
\end{figure}

This is really quite bad. We are using a pretty bad SDF to initialize
the grammar, so maybe that is why. Here is the SDF:

\includegraphics[width=5in]{experiments/2.parsing/shorter_curves/output.d/sdf_8.png}

It is somewhat troubling that it does this badly, though. Let us try
it again with less geometric variation.

\begin{figure}
\caption{On the left, the model curve. On the right, the parsed curve}
\includegraphics[width=6in]{experiments/2.parsing/shorter_curves/output.d/parse_80.png}
\end{figure}

This is basically correct, although the fine details are not very good
looking. This is probably because of the SDF. The shortening rules
only allow the parser to chop off constituents. If the constituents
look bad, then the parse will look bad.

